\documentclass[uplatex]{jsbook}
% \usepackage{multirow}
% \usepackage{amsmath,amssymb}
\usepackage[T1]{fontenc}

\setlength{\textwidth}{\fullwidth}  %本文の幅(textwidth)を全体の幅(=ヘッダ部の幅)にそろえる
\setlength{\evensidemargin}{\oddsidemargin} %偶数ページの余白と奇数ページの余白をそろえる

\title{社会契約の数理モデリングと\\ マルチエージェントシミュレーションによる原理の解明}
% \renewcommand{\thefootnote}{\fnsymbol{footnote}}
\author{Nozomu Miyamoto \\ nontan@sfc.wide.ad.jp}
% \renewcommand{\thefootnote}{\arabic{footnote}}
\date{\today}

\begin{document}

\maketitle

% \twocolumn[
% \begin{@twocolumnfalse}
%   \maketitle
%   \vspace{-6mm}
%   \begin{abstract}
%     ここに概要を書きましょう.
%   \end{abstract}
%   \vspace{2mm}
% \end{@twocolumnfalse}
% ]

% \renewcommand{\thefootnote}{\fnsymbol{footnote}}
% \footnotetext[1]{慶應義塾大学 村井研}
% \renewcommand{\thefootnote}{\arabic{footnote}}

\chapter{序論}
	\section{社会契約とは}
		社会契約とは、法の原理たる概念であり、「なぜ我々の暮らす社会において法が守られるのか」、それを明かすために社会の成り立ちを理論的に説明しようとする試みである。

	\section{自然状態と社会状態}

	\section{ゲーム理論と社会契約}

	\section{これまでの議論の問題点}

	\section{本論の目的と手法}

	\section{本論の構成}

\chapter{商取引ゲーム}
	\section{商取引ゲーム}

	\section{不正防止の不可能性}

\chapter{倫理ある「商取引ゲーム」}
	\section{倫理ある商取引ゲーム}
   
\chapter{約束と評判のゲーム}

\chapter{社会契約過程}

\chapter{結論}

\renewcommand{\refname}{参考文献}
\begin{thebibliography}{数字}
  \bibitem[opt]{key} 文献情報
\end{thebibliography}

\end{document}
